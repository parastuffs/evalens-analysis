\longnewglossaryentry{doyen}
	{name=Doyen}
	{}

\longnewglossaryentry{gestionnaire-campagne}
	{name=Gestionnaire de campagne}
	{S'occupe de la gestion d'une camapagne d'évaluation. Entité pouvant être composée de plusieurs personnes.}

\longnewglossaryentry{superviseur}
	{name=Superviseur}
	{}

\longnewglossaryentry{gestionnaire-com}
	{name=Gestionnaire de communication}
	{Transversal aux facultés.}

\longnewglossaryentry{charge-com-fac}
	{name=Chargé de communication facultaire}
	{Communique différentes informations relatives aux campagnes d'évaluations à la communauté universitaire.}

\longnewglossaryentry{gestionnaire-peda-fac}
	{name=Gestionnaire pédagogique faculaire}
	{Toute personne ayant une responsabilité au niveau faculaire dans le traitement des résultats de campagne, la consultation de ces résultats et la gestion des questionnaires de campagne.}

\longnewglossaryentry{commission-peda-fac}
	{name=Commision pédagogique facultaire}
	{cf. règlement cadre.
	\begin{itemize}
		\item Membre
		\item Président
		Le président siège au conseil des commissions pédagogiques.
	\end{itemize}
	}


\longnewglossaryentry{cocope-lexique}
	{name=Conseil des commissions pédagogiques}
	{Conférence rassemblant les présidents et vice-présidents de chaque commission pédagogique facultaire, ainsi que le recteur, vice-recteur à l'enseignement et deux conseillers pédagogiques désignés par ces derniers.
	Ce conseil est présidé par le recteur ou le vice-recteur qu'il désigne.

	Le CoCoPe a le pouvoir de modifier le règlement cadre et a pour tâche de mettre à jour la banque de questionnaires.}

\newacronym{cocope-acronym}{CoCoPe}{Conseil des Commissions Pédagogiques}

% \longnewglossaryentry{}{CoA}

\longnewglossaryentry{ue}
	{name=Unité d'enseignement}
	{}

\longnewglossaryentry{cours}
	{name=Cours}
	{}

\longnewglossaryentry{etudiant}
	{name=Étudiant}
	{Personne inscrite régulièrement à l'Université Libre de Bruxelles et suivant un certain nombre d'unités d'enseignement.}

\longnewglossaryentry{enseignement}
	{name=Enseignant}
	{Personne dispensant une unité d'enseignement à un ensemble d'étudiants.
	Il peut s'agir d'un professeur ou d'un assitant, titulaire ou non de chacune des unités d'enseignement auxquelles il participe.}

\longnewglossaryentry{titulaire}
	{name=Titulaire}
	{Un enseignant est titulaire d'une unité d'enseignement s'il en est l'enseignant principal.
	Ce statut lui donne accès aux évaluations des enseignants non-titulaires pour les unités d'enseignement qu'ils ont en commun.}

\longnewglossaryentry{cell-com-univ}
	{name=Cellule de communication universitaire}
	{}

\longnewglossaryentry{campagne-eval}
	{name=Campagne d'évaluation}
	{}


\longnewglossaryentry{enquete}
	{name=Enquête}
	{Processus regroupant la prise d'avis des étudiants, le traitement des résultats ainsi que la consultation de ces derniers.}

\longnewglossaryentry{calendrier}
	{name=Calendrier}
	{Ensemble de dates et de périodes liées à une campagne d'évaluation.}

\longnewglossaryentry{periode}
	{name=Période}
	{Intervalle de temps ayant une date et une heure de début et de fin.}

\longnewglossaryentry{dsaa-lexique}
	{name=Département de support aux activités académiques}

\newacronym{dsaa-acronym}{DSAA}{Département de support aux activités académiques}

\longnewglossaryentry{daf-lexique}
	{name=Directeur administratif facultaire}
	{}

\newacronym{daf-acronym}{DAF}{Directeur Administratif Facultaire}

\longnewglossaryentry{pic-lexique}
	{name=Programme individuel de cours}
	{
	\begin{itemize}
		\item Dans le cadre de l'université~: cours suivis par l'étudiant par année d'étude (pré-Paysage, anciennes enquêtes), ou par année académique (post-paysage).
		\item Dans le cadre de la campagne~: cours suivis par l'étudiant lors d'une année académique.
	\end{itemize}
	}

\newacronym{pic-acronym}{PIC}{Programme de Cours Individuel}

\longnewglossaryentry{question-eval}
	{name=Questionnaire d'évaluation}
	{Un questionnaire reprend toutes les questions d'une unité d'enseignement.
	Il est divisé en sections.}

\longnewglossaryentry{reg-reponses}
	{name=Registre des réponses}
	{Registre individuel par utilisateur reprenant les réponses déjà enregistrées.}

\longnewglossaryentry{reg-connexions}
	{name=Registre des connexions}
	{Registre des utilisateurs s'étant connectés à l'application.}

\longnewglossaryentry{reg-activites}
	{name=Registre des activités}
	{Registre de l'activité des utilisateurs dans l'application. (notifications envoyées, etc.)}

% \longnewglossaryentry{}{NRE}
